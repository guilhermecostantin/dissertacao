\chapter[Conclusões]{Conclusões}

Foi identificado, como mostrado na Seção \ref{secao1.2} uma situação-problema nas empresas atualmente, que é a subjetividade na avaliação dos desenvolvedores por parte dos supervisores. Esse problema leva as empresas a terem problemas maiores, de retenção e motivação dos funcionários. Este estudo se propôs então a ajudar as empresas e, se possível fornecer uma ferramenta que auxilia e empresa a enfrentar com mais eficácia esses problemas.

O estudo se baseou em duas principais perguntas. A primeira questionava quais seriam os critérios mais importantes utilizados pelos supervisores ao realizarem sua avaliação sobre os desenvolvedores. Para responder tal pergunta, primeiramente, foi preciso levantar uma série de critérios e optar por aqueles que aparentassem estar mais ligados à realidade das empresas de hoje. Esse levantamento foi realizado utilizando o framework GQM e como resultado foram obtidas 16 métricas, divididas em 4 diferentes grupos, como mostrado na Seção \ref{referencial_levantamento}. Uma vez com as métricas levantadas, foram utilizados algoritmos de seleção de características que ordenaram as 16 métricas por ordem de influência na classe. Foi possível observar ainda intersecções nas métricas mais relevantes da ordenação geral (todas as empresas) com as ordenações individuais das três empresas analisadas, dando mais relevância ao resultado final.

A segunda pergunta questionava se seria possível atingir um classificador de alta acurácia, utilizando os critérios propostos, tanto para o conjunto que reuniam os dados de todas as empresas, quanto para cada empresa que atingisse o mínimo de dados necessários para uma análise individual. Como mostrado no Capítulo 4 e no Capítulo 5, essa pergunta também é considerada como respondida com êxito, visto que foi obtido um classificador com acurácia maior que 85\% para o conjunto de dados de todas as empresas, e para as 3 empresas que se qualificaram para uma análise individual, os melhores classificadores obtiveram uma acurácia de cerca de 80\%, sendo que uma empresa obteve um classificador com uma acurácia de 94\%.

Nesse estudo foi proporcionado então um conjunto de critérios utilizados pelos supervisores de empresas de T.I, para avaliar seus desenvolvedores, além de ordenar esses critérios pela taxa de influência na classificação de importância, provendo uma análise em quais são os fatores mais relevantes. Considerando apenas o conjunto de duas classes criado para otimizar os resultados, as 5 características que mais influenciam a avaliação de importância são:

\begin{itemize}
	\item Pró-atividade
	\item Avaliação de produtividade do desenvolvedor
	\item Capacidade de resolução de problemas complexos
	\item Foco nos resultados
	\item Experiência relevante
\end{itemize}

Levar em consideração essas características no momento da contratação e no acompanhamento individual de cada desenvolvedor tendem formar times de grande importância dentro das organizações.

Além disso, foi criado um classificador de alta acurácia, que pode ajudar, por exemplo, os gerentes de recursos humanos a procurarem candidatos que possuam as características necessárias e que consequentemente possuam um bom potencial de se tornarem parte importante do time, e também pode auxiliar o supervisor a realizar uma avaliação muito mais concreta, saindo da área da subjetividade no momento de avaliar a importância dos seus desenvolvedores. Existe uma taxa de erro inerente ao classificador (no caso da avaliação geral, é de cerca de 15\%), logo, não é recomendável utilizar apenas o classificador sozinho. Ele serve como um instrumento para agregar valor no momento da avaliação do supervisor ou na contratação de um novo funcionário, aliando com uma avaliação qualitativa do desenvolvedor seguindo as características recomendadas nesse mesmo estudo.

Foram obtidos bons resultados com esse estudo sobre a avaliação da importância dos desenvolvedores sob a ótica do supervisor. Porém, entende-se que muito mais pode ser realizado, pois este é um problema significativo e a pesquisa também possui diversas áreas ainda a serem exploradas. Segue uma lista de algumas das possibilidades que para trabalhos futuros:
\begin{enumerate}
	\item Expandir a aplicação da pesquisa, aplicando em mais empresas situadas em diferentes regiões do país e até do mundo, bem como expandir as características utilizadas;
	\item Realizar uma análise qualitativa, em cima dos dados de cada empresa, considerando como a cultura e os valores da empresa influenciam na avaliação de cada critério;
	\item Aplicar esse classificador em colaboradores de repositórios de software \textit{open-source}. Vasilescu et al. \cite{Vasilescu2013} realizou um estudo buscando encontrar relação entre a busca de conhecimento em sites de perguntas e respostas e a produtividade do desenvolvedor, porém utilizou a métrica clássica de produtividade, nesse caso, número de \textit{commits}. Utilizar as métricas levantadas neste estudo como base para validar os resultados ou mesmo detectar e avaliar as diferenças, pode ser uma sugestão de trabalho futuro para ambos os estudos.
\end{enumerate}

Por se tratar de um problema importante para todas as empresas atualmente, entende-se que, além das possibilidades propostas, existem mais inúmeras possibilidades de trabalhos futuros que poderiam usar este estudo como base.
