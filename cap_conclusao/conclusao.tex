\chapter[Conclusões]{Conclusões}

\section{Considerações}
Nós identificamos, como mostrado na seção 1.2 uma situação-problema nas empresas atualmente, que é a subjetividade na avaliação dos desenvolvedores por parte dos supervisores. Esse problema leva as empresas a terem problemas maiores, de retenção e motivação dos funcionários. O nosso estudo se propôs então a ajudar as empresas e, se possível fornecer uma ferramenta que auxilia e empresa a enfrentar com mais eficácia esses problemas.

Seguimos o estudo com base em duas principais perguntas. A primeira questionava quais seriam os critérios mais importantes utilizados pelos supervisores ao realizarem sua avaliação sobre os desenvolvedores. Para isso, primeiramente, precisamos levantar uma série de critérios e optar por aqueles que aparentassem estar mais ligados à realidade das empresas de hoje. Fizemos esse levantamento e chegamos a 16 métricas, divididas em 4 diferentes grupos, como mostrado na seção 2.2 . Utilizamos a ajuda do framework GQM para tal. Uma vez com as métricas levantadas, utilizamos algoritmos de seleção de características que ordenaram as 16 métricas por ordem de influência na classe. Então consideramos que respondemos com sucesso a primeira pergunta realizada.

A segunda pergunta questionava se seria possível atingir um classificador de alta acurácia, utilizando os critérios propostos, tanto para o conjunto que reuniam os dados de todas as empresas, quanto para cada empresa que atingisse o mínimo de dados necessários para uma análise individual. Como mostrado no Capítulo 4 e no Capítulo 5, consideramos também ter respondido com sucesso essa pergunta, visto que conseguimos um classificador  com acurácia maior que 85\% para o conjunto de dados de todas as empresas, e para as 3 empresas que se qualificaram para uma análise individual, os melhores classificadores obtiveram uma acurácia de cerca de 80\%, sendo que uma empresa obteve um classificador com uma acurácia de 94\%.

Nesse estudo nós proporcionamos então um conjunto de critérios utilizados pelos supervisores de empresas de T.I, para avaliar seus desenvolvedores, além de ordenarmos esses critérios pela taxa de influência na classificação de importância, provendo uma análise em quais são os fatores mais relevantes.

Além disso, nós criamos um classificador de alta acurácia, que pode ajudar, por exemplo, os gerentes de recursos humanos a procurarem candidatos que possuam as características necessárias e que consequentemente possuam um bom potencial de se tornarem parte importante do time, e também pode auxiliar o supervisora realizar uma avaliação muito mais concreta, saindo da área da subjetividade no momento de avaliar a importância dos seus desenvolvedores.


\section{Trabalhos Futuros}

Obtivemos bons resultados com esse estudo sobre a avaliação da importância dos desenvolvedores sob a ótica do supervisor. Porém entendemos que muito mais pode ser realizado, pois lidamos com um problema significativo e a pesquisa também possui diversas áreas a serem ainda exploradas. Abaixo listamos algumas das possibilidades que encontramos:
\begin{enumerate}
	\item Expandir a aplicação da pesquisa, aplicando em mais empresas situadas em diferentes regiões do país e até do mundo, bem como expandir as características utilizadas;
	\item Realizar uma análise qualitativa, em cima dos dados de cada empresa, considerando como a cultura e os valores da empresa influenciam na avaliação de cada critério;
	\item Aplicar esse classificador em colaboradores de repositórios de software open-source. Vasilescu et al. [60] realizou um estudo buscando encontrar relação entre a busca de conhecimento em sites de perguntas e respostas e a produtividade do desenvolvedor, porém utilizou a métrica clássica de produtividade, nesse caso, número de commits. Utilizar o nosso estudo as métricas que levantamos como base para validar os resultados ou mesmo detectar e avaliar as diferenças, pode ser uma sugestão de trabalho futuro para ambos os estudos.
\end{enumerate}

Por se tratar de um problema importante para todas as empresas atualmente, entendemos que, além das possibilidades propostas, existem mais inúmeras possibilidades de trabalhos futuros que poderiam usar esse nosso estudo como base.
