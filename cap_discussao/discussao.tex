\chapter[Discussão]{Discussão}

\section{Considerações Iniciais}
O primeiro ponto a ser considerado nessa discussão é a criação do novo conjunto de classes. Como mostrado na Tabela 3, baseado no conjunto original de classes , que possui 5 diferentes classes, nós agrupamos essas 5 classes em apenas 2, criando um novo conjunto de classes que provaram melhorar a performance de todos os algoritmos de classificação que aplicamos, tanto para o conjunto geral de dados, que continha dados de todas as empresas, como para o conjunto gerado pelas três empresas que atingiram o número mínimo para obter uma análise individual. Como pudemos observar na Tabela 4 e na Tabela 5, esse novo conjunto de classes não alterou significativamente a importância dos atributos para a classificação, assim, podemos concluir que essa nova configuração de classes preserva o sentido da classificação original feita pelos supervisores por causa da pequena variação na posição dos atributos na ordenação.
\section{Resultados Gerais (Todas Empresas)}
Vale a pena discutir as 3 primeiras características (mais importantes), que aparecem em ambas as ordenações. É importante mencionar que elas possuem uma correlação positiva com a classe, o que significa que quanto melhor a avaliação da características; melhor a posição na classificação de importância do desenvolvedor. Uma delas é a produtividade do desenvolvedor, sob o ponto de vista do supervisor; nesse caso, produtividade representa a quantidade de trabalho entregue. Isso não foi uma surpresa pois, como mencionamos na introdução desse trabalho, esse é a métrica clássica para avaliação de performance dos desenvolvedores. Por outro lado, as outras duas características trazem novas informações relevantes à discussão.

Capacidade de resolver problemas complexos nos leva à uma direção oposta apresentada pela métrica clássica (quantidade de entregas), porque geralmente faz com que a produção possua uma menor taxa de outputs (LOC ou FP) sobre inputs (recursos, tempo) consumido.

Pró-atividade é na verdade uma característica de comportamento necessária em times envolvidos na solução de problemas complexos invés de sistemas canônicos onde as tarefas são mais previsíveis. Ao avaliarmos os dados utilizando o novo conjunto de classes, Pró-atividade é tida como a característica que mais influencia a avaliação dos supervisores sobre os desenvolvedores. A área de recursos humanos pode conduzir um melhor processo de contratação sabendo que os supervisores dos times de desenvolvimento de software avaliam essas características como requisitos fundamentais.

Sobre o resultado dos classificadores, cada um deles teve um ganho em sua acurácia em torno de 20\% utilizando o novo conjunto de classes. Quando avaliando todas as empresas em conjunto, o que obteve a melhor performance foi o algoritmo NaïveBayes, com uma acurácia de 85,62\%, o que nós consideramos como um resultado útil e de sucesso. O algoritmo J48 também obteve uma acurácia significativa, de 80\%. O uso desses classificadores pode ajudar os supervisores a conduzir uma análise mais coerente do perfil do seu time e de cada desenvolvedor.

\section{Resultados Individuais (Por Empresa)}
Primeiramente, ao falarmos sobre a análise individual que aplicamos a cada empresa, é importante mencionar a distribuição dos desenvolvedores pelas classes de importância. A Empresa A possui uma boa distribuição dos desenvolvedores, tanto para o conjunto de classes original, tanto para o novo conjunto de classes. Não acreditamos que nossa hipótese de que os supervisores possam ter ficados receosos no momento da classificação, que baseia a criação do novo conjunto de classes, se aplique a esse caso.

Diferentemente da Empresa A, as empresas B e C já não possuem uma boa distribuição dos desenvolvedores, por ambos os conjuntos de importância. Acreditamos que a nossa hipótese que baseia a criação do novo conjunto de classes se aplique bem a esses casos, porém mesmo com o agrupamento das classes de importância, obtivemos uma distribuição muito tendenciosa à classe de maior importância (cerca de 80\% dos desenvolvedores dessas empresas são considerados como sendo de “Alta importância”). Dessa forma, como possuímos apenas duas classes no novo conjunto de classes, estatisticamente falando, se o nosso classificador julgasse todos os desenvolvedores como sendo de alta importância, ele teria uma acurácia de cerca de 80\%. Logo, para considerarmos o resultado um sucesso, o classificador, ao utilizar o segundo conjunto de classes, deve apresentar uma acurácia de no mínimo 80\%.

Quando analisamos as características que mais influenciam a avaliação dos supervisores (considerando apenas a ordenação do utilizando o novo conjunto de classes que se provou ser mais eficaz), por cada empresa, notamos algumas grandes diferenças, tanto entre as empresa e em relação ao conjunto geral de dados (todas as empresas). Para todas as empresas, encontramos pelo menos uma característica técnica entre as top três características mais importantes. Nas empresas A e B, foram encontradas características comportamentais (Pró-atividade), de habilidade interpessoal (Comunicação com os colegas) e de compromisso em relação à empresa (Foco no resultado) com melhor posicionamento que a métrica clássica de produtividade (Quantidade de entregas).

Já a Empresa C preza mais por características técnicas, visto que dentre as top 5 primeiras características estão as 4 características técnicas e a métrica clássica de produtividade. Entendemos que essas diferenças nos resultados são devidas às diferenças na cultura e nos valores de cada empresa, vários estudos já foram feito sobre como a cultura corporativa pode interferir na produtividade dos desenvolvedores [13], [14], [18], [42], [57]–[59].

Em relação à classificação, todas as empresas obtiverem um aumento de 10\% a 15\% na acurácia dos classificadores ao utilizar o novo conjunto de classes de importância. A Empresa A, concidentemente, obteve uma acurácia de 79.5\% para ambos os algoritmos J48 e NaïveBayes. A Empresa B, atingiu uma acurácia também igual entre os classificadores, de 80\%, o que não foi considerado um bom resultado devida às condições mencionadas anteriormente sobre a distribuição dos desenvolvedores. A Empresa C por sua vez foi a que obteve a maior acurácia, utilizando o algoritmo NaïveBayes, de 94\%. Devida à sua condição, similar à da Empresa B, consideramos esse resultado um sucesso (o algoritmo J48 também obteve uma acurácia superior ao limite estipulado, de 85\%).

\section{Ameaças À Validade}

Por fim, é importante apontar algumas ameaças à validade desse estudo. O número limitado de desenvolvedores e empresas participantes nesse estudo, e o fato de todas estarem estabelecidas na mesma cidade podem limitar a generalização dos resultados para outros contextos. Apesar disso, observamos várias intersecções no resultado de diferentes empresas, o que pode mitigar parte desse risco. A classificação inicial de importância proporcionado pelos supervisores tendem a serem mais positivas, talvez porque eles não gostariam de dizer que mantém desenvolvedores com baixa importância em seus times. Porém a nova classificação proposta mitiga parte desse risco.

