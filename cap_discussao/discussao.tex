\chapter[Discussão]{Discussão}

\section{Considerações Iniciais}
O primeiro ponto a ser considerado nessa discussão é a criação do novo conjunto de classes. Como mostrado na Tabela 3, baseado no conjunto original de classes, que possui 5 diferentes classes, essas 5 classes foram agrupadas em apenas 2, criando um novo conjunto de classes que provaram melhorar a performance de todos os algoritmos de classificação aplicados, tanto para o conjunto geral de dados, que continha dados de todas as empresas, como para o conjunto gerado pelas três empresas que atingiram o número mínimo para obter uma análise individual. Como é possível observar na Tabela 4 e na Tabela 5, esse novo conjunto de classes não alterou significativamente a importância dos atributos para a classificação, assim, pode-se concluir que essa nova configuração de classes preserva o sentido da classificação original feita pelos supervisores por causa da pequena variação na posição dos atributos na ordenação.

\section{Resultados Gerais (Todas Empresas)}

Vale a pena discutir as 3 primeiras características (mais importantes), que aparecem em ambas as ordenações. são elas:

\begin{itemize}
	\item Pró-atividade
	\item Capacidade de resolução de problemas complexos
	\item Avaliação de produtividade do desenvolvedor
\end{itemize}


Uma delas é a produtividade do desenvolvedor, sob o ponto de vista do supervisor; nesse caso, produtividade representa a quantidade de trabalho entregue. Isso não foi uma surpresa pois, como mencionado na introdução desse trabalho, esse é a métrica clássica para avaliação de performance dos desenvolvedores. Por outro lado, as outras duas características trazem novas informações relevantes à discussão.

Pró-atividade é uma característica comportamental de perfil do desenvolvedor,  e ao se avaliar os dados utilizando o novo conjunto de classes, é tida como a característica que mais influencia a avaliação dos supervisores sobre os desenvolvedores.

Capacidade de resolver problemas complexos no entanto nos leva à uma direção oposta apresentada pela métrica clássica (quantidade de entregas), porque geralmente faz com que a produção possua uma menor taxa de \textit{outputs} (LOC ou FP) sobre \textit{inputs} (recursos, tempo) consumido.

Ambas essas características, pró-atividade e capacidade de resolver problemas, são necessárias em times envolvidos na solução de problemas complexos invés de sistemas canônicos onde as tarefas são mais previsíveis. A área de recursos humanos pode conduzir um melhor processo de contratação sabendo que os supervisores dos times de desenvolvimento de software avaliam essas características como requisitos fundamentais.

Em suma, a métrica clássica de medir a quantidade de entregas continua sendo importante na hora de avaliar a importância dos desenvolvedores, porém, para mitigar os riscos levantados na Seção \ref{secao1.2}, os supervisores devem levar em conta outras características, sejam elas técnicas ou do perfil do desenvolvedor. Um desenvolvedor que possui uma boa capacidade de resolução de problemas complexos, obviamente deve ser alocado para a resolução de tais problemas, e avaliar sua performance pela quantidade de LOC tende a ser um erro, principalmente se for comparar com problemas mais simples. Por outro lado, o supervisor deve estar atento aos desenvolvedores com um perfil mais dinâmico, que buscam tarefas invés de esperar, eles podem, com o acompanhamento apropriado, vir a ser os desenvolvedores que irão bater as metas de produção.

O alinhamento da avaliação de importância pelos supervisores, considerando mais que a métrica clássica, e a busca dessas características técnicas e comportamentais no perfil de candidatos, pelas áreas de contratação das empresas são práticas que podem aumentar a performance de um time / empresa como um todo.

Sobre o resultado dos classificadores, cada um deles teve um ganho em sua acurácia em torno de 20\% utilizando o novo conjunto de classes. Quando avaliando todas as empresas em conjunto, o que obteve a melhor performance foi o algoritmo NaïveBayes, com uma acurácia de 85.62\%, o que pode ser considerado um resultado de sucesso. O algoritmo J48 também obteve uma acurácia significativa, de 80\%. 

O uso desses classificadores pode ajudar os supervisores a conduzir uma análise mais coerente do perfil do seu time e de cada desenvolvedor, auxiliando na implementação da prática mencionada acima, de considerar mais do que a quantidade de entregas para avaliar a importância / performance dos desenvolvedores, e também na evolução do time, de maneira individual e coletiva, trabalhando sobre cada ponto que ainda precisa ser melhorado.

\section{Resultados Individuais (Por Empresa)}


Primeiramente, ao se falar sobre a análise individual aplicada a cada empresa, é importante mencionar a distribuição dos desenvolvedores pelas classes de importância. Ambas as empresas A, B e C, apresentaram uma semelhança em relação à distribuição dos seus desenvolvedores pelas 5 classes originais, que se dá por uma grande concentração de desenvolvedores nas classes mais altas e uma pequena nas classes mais baixas. Esse comportamento reforça a hipótese de que os supervisores possam ter ficados receosos em classificar seus desenvolvedores nas classes mais baixas. Para diminuir esse viés, foi criado esse novo conjunto de 2 classes que agrupa as 3 classes mais baixas em uma nova classe chamada "Baixa importância", e as duas classes mais altas na classe chamada "Alta importância".


Ao serem analisadas as características que mais influenciam a avaliação dos supervisores (considerando apenas a ordenação que utilizou o novo conjunto de 2 classes), por cada empresa, é possível notar algumas grandes diferenças, tanto entre as empresas quanto em relação ao conjunto geral de dados (todas as empresas). Para todas as empresas, encontra-se pelo menos uma característica técnica entre as top três características mais importantes. Nas empresas A e B, foram encontradas características comportamentais (Pró-atividade), de habilidade interpessoal (Comunicação com os colegas) e de compromisso em relação à empresa (Foco no resultado) com melhor posicionamento que a métrica clássica de produtividade (Quantidade de entregas).

Já a Empresa C preza mais por características técnicas, visto que dentre as top 5 primeiras características estão as 4 características técnicas e a métrica clássica de produtividade. É possível atribuir essas diferenças nos resultados às diferenças na cultura e nos valores de cada empresa, vários estudos já foram feitos sobre como a cultura corporativa pode interferir na produtividade dos desenvolvedores \cite{Edmans2011,Jones2000,Scudder1991,AgrellA.andGustafson1994,Guzzo1988,McLean1996,Turcotte2004}.

Em relação à classificação, todas as empresas obtiveram um aumento de 10\% a 15\% na acurácia dos classificadores ao utilizar o novo conjunto de classes de importância. A Empresa A, coincidentemente, obteve uma acurácia de 79.5\% para ambos os algoritmos J48 e NaïveBayes. A Empresa B, atingiu uma acurácia também igual entre os classificadores, de 80\%. Ambas as acurácias máximas das empresas A e B foram menores que a acurácia máxima atingida para o conjunto de dados de todas as empresas. Este não era o resultado, visto que, ao construir classificadores customizados por empresa, esperava-se uma acurácia maior que a geral pela eliminação de diferentes perspectivas dos diferentes supervisores.

A Empresa C por sua vez foi a que obteve a maior acurácia, utilizando o algoritmo NaïveBayes, de 94\%. Esse resultado foi considerado um sucesso, pois supera em quase 10\% a acurácia geral, possuindo uma taxa de erro de apenas 6\%. Além disse, é importante notar que foram agrupados avaliações de 2 supervisores, diferentemente das empresas A e B onde um único supervisor foi responsável pela avaliação. Atribui-se esse alta taxa de acertos à uma cultura bem estabelecida da empresa e a critérios claros de avaliação de importância pelos supervisores da Empresa C.

\section{Ameaças À Validade}

Por fim, é importante apontar algumas ameaças à validade desse estudo. O número limitado de desenvolvedores e empresas participantes nesse estudo, e o fato de todas estarem estabelecidas na mesma cidade podem limitar a generalização dos resultados para outros contextos. A variabilidade na cultura das empresas participantes e nos critérios que cada supervisor usa para avaliar seus desenvolvedores também podem impactar o resultado geral. Apesar disso, é possível observar várias intersecções no resultado de diferentes empresas, e das empresas com o resultado geral, o que pode mitigar parte desse risco. A classificação inicial de importância proporcionado pelos supervisores tendem a serem mais positivas, talvez porque eles não gostariam de dizer que mantém desenvolvedores com baixa importância em seus times. Porém a nova classificação proposta mitiga parte desse risco.

A possível ausência de fatores para a avaliação dos supervisores também é uma ameaça, porém o levantamento de fatores baseou-se em vários estudos, que também buscavam os fatores de maior relevância na avaliação de desenvolvedores \autopageref{tabela_referencias}, e também houve uma preocupação em selecionar características distintas, não apenas técnicas ou comportamentais, diminuindo esse risco.