\chapter[Introdução]{Introdução}

\section{Considerações Iniciais}\label{secao1.1}
Atualmente, as empresas em geral estão investindo em técnicas para aumento de produtividade, com o intuito de aumentar a competitividade no mercado, e isso não é diferente para a indústria de software, que permanece investindo em métodos, ferramentas e melhores práticas para ter um ganho na produção de seu software \cite{deBarrosSampaio2010}.
Porém, diferentemente do hardware, que tem ganhos em ordens de magnitude de preço e performance por década, a produção de software parece ter dificuldade em evoluir \cite{Boehm1987}. As taxas de produtividade atuais são similares às taxas de décadas atrás (uma ou duas linhas de código por homem-hora)\cite{Boehm1987}. Brooks et al. \cite{BrooksJr1987} afirma ainda que não há nenhuma técnica, seja de tecnologia ou de gerenciamento, que por si só prometa um aumento de uma ordem de magnitude na produtividade, simplicidade e confiabilidade do software.

Os métodos tradicionais para se medir produtividade em desenvolvimento de software são baseados em linhas de código (LOC) e pontos de função (FP)\cite{Wagner2008}, por exemplo, a quantidade de LOC ou FP produzidos por um desenvolvedor em uma hora. Uma definição um pouco mais abstrata coloca produtividade como sendo os outputs entregues pelos inputs consumidos, podendo os outputs ser LOC, FP ou alguma outra saída considerada relevante, e inputs como recursos utilizados para produzir aquela saída (tempo, pessoal, etc), como mostrado na Equação 1 \cite{Boehm1987, Walston1977, Yu1991}.

\begin{equation}
\text{Produtividade} = \dfrac{\text{outputs produzidos pelo processo}}{\text{inputs consumidos pelo processo}}
\end{equation} 

\section{Problema e Motivação}\label{secao1.2}
Dentro das empresas de TI, geralmente é encontrado um departamento de recursos humanos ou desenvolvimento pessoal. Esses departamentos são responsáveis pela devida remuneração e determinação de cargos e posições dentro das empresa, e geralmente o fazem com base em avaliações de desempenho vinda dos supervisores responsáveis de cada área. Geralmente, encontramos um ou mais supervisores ou gerentes responsáveis na área de desenvolvimento, comumente conhecida pelos nomes de indústria ou fábrica de software.

Utilizar apenas o método tradicional citado na seção \ref{secao1.1} para avaliar o desempenho de um desenvolvedor pode ser muito negativo para a empresa. Esse método de avaliar apenas LOC por exemplo é considerado primitivo e produz resultados que não estão de acordo com a realidade \cite{Symons2010}. Quantidade de linhas de código não leva em consideração o esforço necessário para escreve-las, nem o conhecimento que serviu de base para o mesmo acontecer. Problemas complexos, por exemplo, geralmente necessitam de um desenvolvedor mais experiente para serem resolvidos, e muitas vezes, não são necessários muitas linhas de código para o fazer.

Existem outras noções de produtividade que também não são levadas em consideração ao avaliar apenas linhas de código, por exemplo, um desenvolvedor com conhecimento em uma ferramenta muito específica, ou o supervisor, podem ser consultados frequentemente por outros desenvolvedores, agilizando e melhorando o processo de desenvolvimento de outros desenvolvedores, logo, esse desenvolvedor/supervisor possui uma produtividade indireta, que não é mensurada no método tradicional pois quando se está ajudando e ensinando, não se está escrevendo linhas de código.

Características comportamentais , encontradas no perfil do desenvolvedor, também podem influenciar na avaliação do desenvolvedor, aumentando sua relevância para a empresa. Pró-atividade, criatividade e liderança são alguns exemplos de características que geralmente fazem diferença na carreira de um funcionário dentro de uma organização. O método clássico também falha em não considerar esses quesitos, pois a avaliação de linhas de código não demonstram inovação, alinhamento com o negócio, etc.

Vários estudos se dedicam a levantar fatores que influenciam na produtividade, tanto no desenvolvimento quanto na manutenção de software, \cite{deBarrosSampaio2010, Wagner2008, Calow1991, Vosburgh1984}, dentre outros. Entender esses fatores e ter um mecanismo de avaliação de produtividade justo é muito importante para as empresas que desenvolvem qualquer tipo de software.

Retenção de talentos \cite{Boehm2000, Chatzoglou1997, DeMarco1987, Guzzo1988, Scudder1991, Wohlin1995, Wohlin2001} e motivação da equipe \cite{Boehm1987, DeMarco1987, Boehm1984, Jones2000, Sharp2009, Boehm1982, Boehm1988, Hantos2000}, por exemplo, são duas questões de extrema relevância para qualquer time. Motivação é tido como um dos fatores chave para o sucesso do software \cite{Sharp2009}, e software é feito por pessoas, e pessoas, quando tem seu trabalho reconhecido e valorizado tendem a produzir mais e melhor. Uma avaliação de desempenho que considere apenas um aspecto, como a quantidade de entregas, e não leva em conta a dificuldade e a finalidade do código, e o relacionamento do dia a dia com os colegas e com a empresa, não é uma avaliação justa, que faz com que ocorra a desmotivação individual ou geral da equipe. Rotatividade de funcionários é um problema comum à empresas de software\cite{Abdel-Hamid1991, Wallace2004}, e uma alta taxa de rotatividade pode levar à uma consequente diminuição de produtividade devida a uma perda de conhecimento \cite{Melo2011, Coram2005}, além do aumento de custo com contratação e treinamento e o mais importante, a perda de talentos que vão em busca de reconhecimento em outro empresa (até mesmo no concorrente).

Várias empresas estão começando a ganhar consciência dessas questões e estão motivadas a melhorar a forma que são feitas as avaliações de desempenho dos desenvolvedores. Esse trabalho visa investigar como os supervisores entendem a noção de importância, indicando quais fatores são mais relevantes em suas avaliações sobre os desenvolvedores. Estamos procurando responder essas perguntas:

\begin{enumerate}
	\item Quais são os critérios mais importantes usados pelos supervisores em sua classificação sobre os desenvolvedores?
	\item É possível construir um classificador de desenvolvedores de alta acurácia, usando os critérios propostos? Essa pergunta pode ser refinada em duas novas perguntas:
	\begin{enumerate}
		\item É possível se obter um classificador genérico, i.e., independente de empresa?
		\item É possível se obter classificadores customizados para cada empresa?
	\end{enumerate}
\end{enumerate}



\section{Objetivos e Contribuição}
Como mencionado na seção \ref{secao1.1}, o desempenho de um desenvolvedor é muito relacionado com sua produtividade, que possui um conceito clássico de quantidade de entregas. Porém, os supervisores e gerentes que participam ativamente da sua área de desenvolvimento e convivem com os desenvolvedores que ali trabalham, possuem percepções diferentes sobre cada membro de seu time. Isso nada mais é que uma avaliação subjetiva de desempenho, que não considera apenas a produção individual resultante, mas diversas outras características que fazem que o desenvolvedor possua uma determinada importância na visão do seu supervisor.

Entendemos que a métrica mais comum hoje é a “produtividade” no sentido de quantidade de entregas. Porém, sabemos que muito mais é levado em consideração ao avaliar a importância de um determinado desenvolvedor, o problema é que essa avaliação geralmente acontece de uma forma subjetiva por parte do supervisor, isto é, cada supervisor, de acordo com a vivência, identifica quais os desenvolvedores mais importantes devido à sua percepção, sem a orientação de métricas bem estabelecidas.

Fizemos um levantamento, baseado em estudos passados, de várias métricas que podem influenciar na avaliação de cada desenvolvedor. Realizaremos uma pesquisa com sujeitos humanos (supervisores de áreas de desenvolvimento representando empresas ou times), onde cada um desses sujeitos humanos irá avaliar individualmente os desenvolvedores seguindo as métricas levantadas. Iremos então analisar o conjunto de dados resultante da aplicação da pesquisa com intuito de identificar um padrão nas avaliações dos desenvolvedores. Dessa forma, pretendemos apontar quais as métricas que mais influenciam na avaliação dos supervisores sobre os desenvolvedores, e também obter um classificador de importância dos desenvolvedores, utilizando essas métricas mais relevantes, que esperamos seja de alta acurácia. Os supervisores que preencherem uma quantidade significativa de dados (10 ou mais questionários) também terão sua empresa/time analisados individualmente, tanto na descoberta das métricas mais relevantes como na construção do classificador de importância do desenvolvedor.


\section{Estrutura da Dissertação}
Além do presente capítulo introdutório, esta pesquisa apresenta-se desenvolvida e documentada dentro da seguinte estrutura organizacional:

\begin{description}
	\item[Capítulo 2: Referencial Teórico] \hfill \\
	Nesse capítulo apresentaremos os conceitos importantes que serão abordados ao longo desse trabalho, bem como outros estudos que atuaram em cima desse mesmo tópico e que serviram de base para produção desse estudo.
	
	\item [Capítulo 3: Metodologia] \hfill \\
	Nesse capítulo será apresentado a metodologia usada para conduzir a pesquisa com sujeito humanos, mostrando a criação do conjunto de critérios para os supervisores avaliarem os desenvolvedores e a aplicação da pesquisa. Será mostrado também a estratégia de seleção de características utilizada para identificar quais critérios possuem maior relevância no momento da avaliação do supervisor, e a estratégia para a construção do classificador  de importância dos desenvolvedores.
	
	\item[Capítulo 4: Resultados Gerais (todas empresas) ] \hfill \\
	Nesse capítulo apresentaremos o resultado da aplicação da pesquisa, mostrando o número de supervisores e empresas envolvidos na pesquisa e a quantidade de avaliações de desenvolvedores obtidas. Mostraremos também o resultado da aplicação dos algoritmos de seleção de características e de classificação para todas as empresas.
	
	\item[Capítulo 5: Resultados individuais (por empresa) ] \hfill \\
	Nesse capítulo apresentaremos o resultado da aplicação da pesquisa, aplicação dos algoritmos de seleção de características e de classificação para as empresas que atingirem o número mínimo de avaliações de desenvolvedores.
	
	\item[Capítulo 6: Discussão ] \hfill \\
	Aqui discutiremos os resultados obtidos da aplicação dos algoritmos de seleção de características e classificação, apresentando também algumas ameaças à validade desse estudo.
	
	\item[Capítulo 7: Conclusões ] \hfill \\
	Por fim, baseada nos resultados obtidos, apresentaremos a conclusão do trabalho e levantaremos algumas possibilidades de trabalhos futuros que possam ter como base esse estudo.
	
\end{description}

