\chapter[Referencial Teórico]{Referencial Teórico}

Este estudo gira em torno do conceito de importância dos desenvolvedores, sob as perspectivas dos seus supervisores. Porém, notamos que esse conceito se mistura com conceitos de produtividade, performance ou eficiência. Vários estudos, que serviram de base para levantarmos as métricas para os supervisores avaliarem os desenvolvedores, eram na verdade estudos sobre produtividade e controle de custos e qualidade de software.

Ao longo do tempo, vários estudos se dedicaram a encontrar e classificar os chamados fatores de produtividade \cite{Vosburgh1984, Walston1977,Brooks1981,Hanson1985,Jones1986,Boehm1988,Jones1997,Scudder1991,Banker1991,Boehm1984,Banker1987,Scacchi1995,Briand1998,Jones2000,Lokan2001,Clincy2003,Wagner2008,deBarrosSampaio2010}. Em COCOMO \cite{Boehm2000} (Constructive Cost Model) considerado um dos estudos mais proeminentes na área, Boehm classifica os fatores de produtividade diversas categorias (produto, projeto, etc.). Abstraindo ainda mais as categorias, ele classifica os fatores em fatores técnicos e fatores do tipo “soft” \cite{Wagner2008}, que são os fatores não-técnicos que influenciam na produtividade.

De Marco e Lister \cite{DeMarco1987} apontaram que “talvez os maiores problemas de trabalhar com sistemas não são tanto tecnológicos quanto sociológicos”. Em seu estudo, por exemplo, eles afirmam que um dos fatores que mais influencia na produtividade é a rotatividade de funcionários (como ressaltado na Seção \ref{secao1.2}). Em suma, eles proporcionaram o primeiro e mais compreensível estudo sobre os fatores “soft” influenciando na produtividade dos desenvolvedores de software \cite{Wagner2008}.

Como nosso trabalha se dedica a medir a importância do desenvolvedor, focaremos mais nos fatores do tipo “soft” do que nos fatores do tipo técnico (que envolvem fatores relativos à produto e projeto, como tamanho do software, levantamentos de requisitos , etc \cite{deBarrosSampaio2010}).

\section{Levantamento de Características}\label{referencial_levantamento}
Nessa seção mostraremos então quais os fatores do tipo “soft” utilizaremos para traduzir do abstrato para o concreto a avaliação dos supervisores sobre seus desenvolvedores. Mostraremos também estudos onde esses fatores foram referenciados, sob o contexto de avaliação de fatores que influenciam na produtividade dos desenvolvedores de software. Esses fatores, como mostrado na Seção \ref{secao3.2.1}, serão utilizados para compor as métricas do nosso modelo Goal-Question-Metric (GQM), que nos auxiliará na condução da nossa pesquisa com as empresas. 

Primeiramente, é importante citar que agrupamos os fatores levantados por semelhança de conceito de forma a deixar o levantamento mais conciso. Esses grupos ajudarão na elaboração das perguntas do modelo GQM. São eles:


\begin{description}
	\item[Características técnicas] \hfill \\
	Aqui agrupamos os fatores relativos às habilidades do desenvolvedor. Dentre os fatores mais citados na literatura, selecionamos quatro, que contemplam a experiência do desenvolvedor em um determinado método ou ferramenta, se ele possui conhecimento especializado em uma determinada tecnologia, ou mesmo se possui uma diversidade de conhecimentos em variadas tecnologias, e sua capacidade em resolver problemas complexos.
	
	\item[Características comportamentais (quando em equipe)] \hfill \\
	Encontramos diversos estudos que mostram como a coesão e a comunicação de um time de desenvolvimento influencia na produtividade dos desenvolvedores pertencentes a esse time. Dividimos as métricas então sobre o comportamento do desenvolvedor quando encontra um problema (se ele é do tipo introspectivo que tenta resolver sozinho ou comunicativo que logo consulta alguém do time), quando algum colega solicita ajuda e de um modo geral como é a sua comunicação com seus colegas de time.
	
	\item[Características Individuais (encontradas no perfil do desenvolvedor)] \hfill \\
	Aqui entramos um pouco na ciência de Gestão de Pessoas. Para usar como fatores de importância do desenvolvedor, mapeamos algumas das competências citadas como mais importantes para as organizações hoje em dia. Competência, segundo Chiavenato \cite{Chiavenato2008}, constitui um repertório de comportamentos capazes de integrar, mobilizar, transferir conhecimentos, habilidades, julgamentos e atitudes que agregam valor econômico à organização. As competências selecionadas são mostradas na \autoref{tabela1} e os estudos que as referenciam geralmente estão ligados a estudos de gestão por competências.
	
	\item[Compromisso com o Time/Empresa] \hfill \\
	Amplamente encontrado em literatura relativa às metodologias de desenvolvimento ágil, por serem fatores que baseiam a filosofia ágil que visa melhorar a produtividade de um time, estão os fatores mapeados à essa categoria. Foco nos clientes, nos resultados e organização são alguns deles. Outro fator também se encaixa bem à essa categoria, que é o tempo de trabalho do desenvolvedor na empresa, que representa a fidelização do funcionário.
	
\end{description}

\begin{table}[h]
	\caption{Levantamento dos fatores de importância por grupo de semelhança}
	\label{tabela1}
	\def\arraystretch{2}
	\begin{tabular}{|p{4cm}|p{8cm}|>{\centering\arraybackslash}p{2.5cm}|}
		\hline
		\textbf{Agrupamentos} & \textbf{Fatores de importância} & \textbf{Referências (\autoref{tabela_referencias})} 
		\\ \hline
		
		\multirow{4}{*}{\parbox{4cm}{Características \\técnicas}} & Experiência relevante & \multirow{4}{*}{(1)}
		\\ \cline{2-2} & Conhecimento especializado & 
		\\ \cline{2-2} & Diversidade de habilidades & 
		\\ \cline{2-2} & Capacidade de resolução de problemas complexos & 
		\\ \hline
		
		\multirow{4}{*}{\parbox{4cm}{Características \\comportamentais}} & Principal comportamento do desenvolvedor ao encontrar um problema & \multirow{4}{*}{(2)}
		\\ \cline{2-2} & Disposição para ajudar colegas quando solicitado & 
		\\ \cline{2-2} & Comunicação com os colegas & 
		\\ \hline
		
		\multirow{4}{*}{\parbox{4cm}{Características \\individuais}} & Liderança & \multirow{4}{*}{(3)}
		\\ \cline{2-2} & Pró-atividade & 
		\\ \cline{2-2} & Criatividade & 
		\\ \cline{2-2} & Empreendedorismo & 
		\\ \hline
		
		\multirow{4}{*}{\parbox{4cm}{Compromisso com \\o Time/Empresa}}  & Foco no cliente & \multirow{4}{*}{(4)}
		\\ \cline{2-2} & Foco nos resultados & 
		\\ \cline{2-2} & Organização e planejamento & 
		\\ \cline{2-2} & Tempo de trabalho & 
		\\ \hline
	\end{tabular}
\end{table}

\begin{table}[h]
	\caption{Revisão da literatura correlata aos fatores de importância}
	\label{tabela_referencias}
	\def\arraystretch{2}

	\begin{tabular}{|>{\centering\arraybackslash}p{2.5cm}|p{12.5cm}|}
		\hline
		\textbf{Grupos de fatores} & \textbf{Estudos onde os fatores são citados}                                                                                                                                                      \\ \hline
		1                                                     & {\parbox[c][4.5cm][c]{12.5cm}{\cite{Chatzoglou1997,Cole1995,Jones1986,Maxwell2000,Banker1991,Boehm2000,Brooks1981,Finnie1993,Jones2000,Lakhanpal1993,Scudder1991,Turcotte2004,Vosburgh1984,Walston1977,Wohlin1995,Wohlin2001}}} \\ \hline
		2                                                     & {\parbox[c][3cm][c]{12.5cm}{\cite{Alper2000,Boehm2000,Chatzoglou1997,Lakhanpal1993,Rasch1991,Scudder1991,Vosburgh1984,Walston1977,Wohlin1995,Lalsing2012}}}                                                                   \\ \hline
		3                                                     & {\parbox[c][2cm][c]{12.5cm}{\cite{Chiavenato2008,Lalsing2012,FariaSueli2005,Dutra2004,Fleury2001}}}                                                                                                                         \\ \hline
		4                                                     & {\parbox[c][2cm][c]{12.5cm}{\cite{Lalsing2012,Melo2011,FariaSueli2005,Schwaber2004,Coram2005}}}                                                                                                                               \\ \hline
	\end{tabular}
\end{table}

\section{Data Mining e Classificação}
Segundo Witten \cite{Holmes},  Mineração de dados  se trata de coletar os dados brutos e transformá-los em algo mais útil, como informação ou predições de algo que pode vir a acontecer, que podem ser úteis no mundo real.

Para realizarmos as análises deste estudo, utilizaremos uma ferramenta chama WEKA (Waikato Environment for Knowledge Analysis)\cite{Holmes}. Weka é um software open-source que contém um grande número de algoritmos para classificação, pré-processamento de dados, seleção de características, clusterização, regras de associação, etc. Faremos uso desses algoritmos de aprendizado de máquina, para realizar a mineração de dados no conjunto de dados resultante da nossa pesquisa.

O uso de algoritmos de aprendizado de máquinas não é novidade no estudo sobre a produtividade dos desenvolvedores. Sharpe et al. \cite{Sharpe2005} utilizou algoritmos de reconhecimento de padrões, mais especificamente, algoritmos de clusterização, para classificar os desenvolvedores de acordo com sua produtividade. No nosso caso, como iremos obter a avaliação dos desenvolvedores pelos supervisores em cima de um conjunto pré-definido de classes, utilizaremos algoritmos de seleção de características para podermos selecionar apenas as características mais relevantes nessa avaliação e algoritmos de classificação que utilizarão essa avaliação dos supervisores como base (classe) para tentar obter a maior acurácia possível perante essa classificação prévia feita pelos supervisores.


