\chapter[Resultados Gerais]{Resultados Gerais (Todas Empresas)}\label{resultados_gerais}

Seguindo os passos apresentados no capítulo anterior, serão mostrados os resultados da aplicação da pesquisa, a seleção de características executada no conjunto de dados gerados por essa pesquisa e a aplicação dos classificadores e suas respectivas acurácias na classificação dos desenvolvedores.

\section{Pesquisa}\label{secao4.2}
Primeiramente serão apresentados os dados coletados com a pesquisa. Onze respondentes (supervisores) forneceram 61 respostas (avaliações únicas de desenvolvedores). Em alguns casos, mais de um supervisor trabalham na mesma empresa, porém eles supervisionam diferente times. No total, oito empresas foram envolvidas na coleta de dados. Na tabela \ref{tabela2_1}, podemos ver, por empresa, quantos supervisores (que supervisionam diferentes times) responderam a pesquisa e quantos desenvolvedores foram avaliados.

Assim como mostrado na Seção \ref{caracterizacao_respondentes}, as empresas participantes são empresas que possuem uma área de desenvolvimento de software com ao menos um supervisor. Todas as empresas estão situadas na cidade de Uberlândia e trabalham em seus próprios produtos (não apenas desenvolvem software de maneira terceirizada), mas elas variam em tamanho, considerando quantidade de desenvolvedores como critério, setor de operação (ERP, Telecomunicações, etc), e podem variar em tecnologia utilizada.


\begin{table}[h]
	\caption{Relação dos participantes na pesquisa}
	\label{tabela2_1}
	\def\arraystretch{2}
	\begin{tabular}{|>{\centering\arraybackslash}p{3cm}|>{\centering\arraybackslash}p{5.75cm}|>{\centering\arraybackslash}p{5.75cm}|}
		\hline
		{\textbf{Empresa}} &
		\parbox{5.75cm}{\textbf{Quantidade de supervisores}} & 
		\parbox{5.75cm}{\textbf{Quantidade de desenvolvedores avaliados}} \\ \hline
		A                                      & 2                                                              & 20                                                                    \\ \hline
		B                                      & 1                                                              & 10                                                                    \\ \hline
		C                                      & 3                                                              & 20                                                                    \\ \hline
		D                                      & 1                                                              & 4                                                                     \\ \hline
		E                                      & 1                                                              & 3                                                                     \\ \hline
		F                                      & 1                                                              & 2                                                                     \\ \hline
		G                                      & 1                                                              & 1                                                                     \\ \hline
		H                                      & 1                                                              & 1                                                                     \\ \hline
		\textbf{total}                         & 11                                                             & 61                                                                    \\ \hline
	\end{tabular}
\end{table}

Foi pedido para os supervisores classificarem os desenvolvedores em 5 graus de importância. Esses graus de importância e a distribuição dos desenvolvedores avaliados neles é mostrado na \autoref{fig_3}.

Analisando os resultados da pesquisa, percebe-se que os supervisores podem ter sido, em certo nível, conservadores em classificar seus desenvolvedores nas classificações mais baixas de importância. Uma possível explicação para esse comportamento é o viés criado ao seguir a linha de raciocínio de que se o desenvolvedor possui baixa importância, ele não estaria na equipe.
Esse viés na classificação dos desenvolvedores impacta diretamente a acurácia dos classificadores, visto que eles usam a classe como base para tentar descobrir um padrão nos dados. Caso a classificação não tenha ficado clara para todos os respondentes, os algoritmos podem se confundir e apresentar erros leves, classificando um determinado indivíduo em classes similares, mas não à classe que ele realmente pertence. Esses erros em grande quantidade interferem e possuem um impacto negativo significativo na acurácia do classificador que é proposto neste estudo.

A partir dessa análise, juntamente com a análise da matriz de confusão gerada pela aplicação dos algoritmos descritos na Seção \ref{secao3.4}, decidiu-se agrupar os desenvolvedores em apenas duas classes, baseadas nas cinco classes originais de importância, como mostrado na \autoref{tabela3}, com o intuito de obter uma análise mais realística.

Utilizando esse novo conjunto de classes para agrupar os desenvolvedores, obteve-se a distribuição mostrada na \autoref{fig_4}.

%inserir tabela 3

\begin{table}[h]
	\centering
	\caption{Novo conjunto de classes de desenvolvedores}
	\label{tabela3}
	\def\arraystretch{1.5}
	\begin{tabular}{|p{6cm}|p{8.5cm}|}
		\hline
		\textbf{Novo conjunto de classes}  & \textbf{Conjunto de classes originais} \\ \hline
		\multirow{2}{*}{Alta importância}  & Muito importante                       \\ \cline{2-2} 
		& Importante                             \\ \hline
		\multirow{3}{*}{Baixa importância} & Importância média                      \\ \cline{2-2} 
		& Pouco importante                       \\ \cline{2-2} 
		& Muito pouco importante                 \\ \hline
	\end{tabular}
\end{table}



\begin{figure}[h]
	\centering
	\includegraphics[scale=0.45]{figs/geral/imagem-classe-original}
	\caption{\label{fig_3}Distribuição dos desenvolvedores avaliados pelas classes de importância}
\end{figure}

\begin{figure}[h]
	\centering
	\includegraphics[scale=0.45]{figs/geral/imagem-classe-alternativa}
	\caption{\label{fig_4}Distribuição dos desenvolvedores avaliados pela nova classe de importância}
\end{figure}

\section{Seleção de Características}\label{secao4.3}

Como explicado na Seção \ref{secao3.3}, foi utilizado o algoritmo \textit{GainRatio} para ordenar as características propostas, para prosseguir com a seleção de características. A \autoref{tabela4} apresentam as características, a posição média de cada característica na ordenação e o mérito médio (grau de influência da característica na classificação, calculado pelo algoritmo) resultante da aplicação do algoritmo usando as classes originais e a \autoref{tabela5} mostra a mesma visão, utilizando o novo conjunto de 2 classes (são apresentadas as médias da posição e do mérito pois o algoritmo utiliza o \textit{10-fold cross-validation}, ou seja, ele roda 10 vezes, selecionando diferentes conjuntos de dados, e retorna a média dos resultados).

É possível notar algumas semelhanças e diferenças entre essas tabelas. as três primeiras características da primeira tabela, por exemplo, ocupam também as três primeiras posições na segunda, apesar de estarem em ordem diferente, o que mostra que a mudança do conjunto de classes não teve um impacto muito significativo na relevância dessas características. Porém, após as três primeiras, é possível notar algumas diferenças mais expressivas no posicionamento dos características, mostrando então um maior impacto nas características menos relevantes causada pela troca do conjunto de classes.

%inserir Tabela 4
\begin{table}[h]
	\caption{Ordenação das características (conjunto original de 5 classes)}
	\label{tabela4}
	\def\arraystretch{2}
	\begin{tabular}{|p{8.5cm}|>{\centering\arraybackslash}p{3cm}|>{\centering\arraybackslash}p{3cm}|}
		\hline
		\textbf{Características}                                                      & \textbf{Posição média} & \textbf{Mérito médio} \\ \hline
		Capacidade de resolução de problemas complexos                          & 1.4 +- 0.49                                 & 0.303 +- 0.03                                                                          \\ \hline
		Qual a sua avaliação sobre a produtividade do desenvolvedor em questão? & 1.6 +- 0.49                                 & 0.29  +- 0.022                                                                         \\ \hline
		Pró-atividade                                                           & 3.6 +- 0.92                                 & 0.226 +- 0.01                                                                          \\ \hline
		Experiência relevante                                                   & 4.8 +- 1.17                                 & 0.211 +- 0.014                                                                         \\ \hline
		Diversidade de habilidades                                              & 5.6 +- 1.36                                 & 0.202 +- 0.022                                                                         \\ \hline
		Conhecimento especializado                                              & 6.1 +- 2.51                                 & 0.2   +- 0.026                                                                         \\ \hline
		Tempo de trabalho (meses)                                               & 7.1 +- 1.37                                 & 0.184 +- 0.012                                                                         \\ \hline
		Criatividade                                                            & 7.4 +- 1.62                                 & 0.18  +- 0.021                                                                         \\ \hline
		Foco nos resultados                                                     & 8.7 +- 1.27                                 & 0.167 +- 0.017                                                                         \\ \hline
		Foco no cliente                                                         & 10 +- 2.45                                  & 0.148 +- 0.023                                                                         \\ \hline
		Principal comportamento do desenvolvedor                                & 11.1 +- 1.58                                & 0.14  +- 0.021                                                                         \\ \hline
		Comunicação com os colegas                                              & 11.6 +- 1.02                                & 0.137 +- 0.011                                                                         \\ \hline
		Organização e planejamento                                              & 12.3 +- 1.27                                & 0.122 +- 0.015                                                                         \\ \hline
		Liderança                                                               & 14.5 +- 1.2                                 & 0.097 +- 0.018                                                                         \\ \hline
		Empreendedorismo                                                        & 15 +- 0.89                                  & 0.087 +- 0.012                                                                         \\ \hline
		Disposição para ajudar colegas quando solicitado                        & 15.2 +- 0.6                                 & 0.084 +- 0.014                                                                         \\ \hline
	\end{tabular}
\end{table}
\clearpage
%inserir Tabela 5

\begin{table}[h]
	\caption{Ordenação das características (novo conjunto de 2 classes)}
	\label{tabela5}
	\def\arraystretch{2}
	\begin{tabular}{|p{8.5cm}|>{\centering\arraybackslash}p{3cm}|>{\centering\arraybackslash}p{3cm}|}
		\hline
		\textbf{Características}                                                      & \textbf{Posição média} & \textbf{Mérito médio} \\ \hline
		Pró-atividade                                                           & 1.2 +- 0.4             & 0.168 +- 0.01         \\ \hline
		Qual a sua avaliação sobre a produtividade do desenvolvedor em questão? & 1.9 +- 0.54            & 0.156 +- 0.017        \\ \hline
		Capacidade de resolução de problemas complexos                          & 3.2 +- 0.6             & 0.126 +- 0.013        \\ \hline
		Foco nos resultados                                                     & 4.8 +- 0.98            & 0.112 +- 0.018        \\ \hline
		Experiência relevante                                                   & 4.9 +- 1.3             & 0.107 +- 0.018        \\ \hline
		Criatividade                                                            & 6.5 +- 1.36            & 0.095 +- 0.008        \\ \hline
		Organização e planejamento                                              & 6.9 +- 2.21            & 0.09 +- 0.014         \\ \hline
		Diversidade de habilidades                                              & 8.1 +- 1.14            & 0.08 +- 0.011         \\ \hline
		Conhecimento especializado                                              & 9.2 +- 1.78            & 0.071 +- 0.013        \\ \hline
		Foco no cliente                                                         & 10.3 +- 1.95           & 0.063 +- 0.012        \\ \hline
		Tempo de trabalho (meses)                                               & 10.8 +- 1.08           & 0.063 +- 0.01         \\ \hline
		Disposição para ajudar colegas quando solicitado                        & 11.6 +- 2.24           & 0.057 +- 0.019        \\ \hline
		Liderança                                                               & 12 +- 0.89             & 0.052 +- 0.004        \\ \hline
		Comunicação com os colegas                                              & 13.6 +- 0.92           & 0.039 +- 0.009        \\ \hline
		Empreendedorismo                                                        & 15.5 +- 0.5            & 0.015 +- 0.005        \\ \hline
		Principal comportamento do desenvolvedor                                & 15.5 +- 0.5            & 0.016 +- 0.007        \\ \hline
	\end{tabular}
\end{table}
\clearpage

\section{Classificação}\label{secao4.4}

Considerando que as características foram ordenados, foi aplicado a técnica de seleção de características e usado apenas as características mais relevantes na classificação. Para determinar quantas características precisam ser selecionadas para se obter uma maior performance, foi conduzido um teste exaustivo (os classificadores rodaram com um crescente número de características selecionadas, de 2 a 16) e escolhido a configuração com melhor performance (8 características). 

A \autoref{tabela7.1} e a \autoref{tabela7.2} mostram a aplicação exaustiva do algoritmo que associa a seleção de características com os algoritmos de classificação, para o J48 e para o NaïveBayes respectivamente, mostrando na primeira coluna a quantidade de características utilizadas naquele momento, e na segunda e terceira coluna a porcentagem de acertos (percentual de instâncias corretamente classificadas) para o conjunto original de 5 classes e para o conjunto de 2 classes, respectivamente.

A \autoref{tabela6} mostra os resultados da aplicação do algoritmo J48.Neste caso, o J48 não apresentou uma boa performance, com acurácia perto de 60\%, para o conjunto original de classes, porém já apresentou uma acurácia significativa ao ser aplicado sobre o novo conjunto de classes (80\% de acertos). No caso do J48, para ambos os conjuntos de classe, o classificador obteve a melhor performance utilizando apenas 2 características, como mostrado na \autoref{tabela7.1}.

NaïveBayes, como mostrado na \autoref{tabela7} obteve uma performance similar ao J48 quando aplicado no conjunto original de classes (acurácia perto de 61\%). Já para o novo conjunto de classes, NaïveBayes atingiu uma acurácia expressiva de 85\%. Ambos os melhores resultados do NaïveBayes foram atingidos utilizando 8 características, como mostra a \autoref{tabela7.2}.

%inserir Tabela 6

\begin{table}[h]
	\centering
	\caption{Aplicação do J48 para os diferentes conjuntos de classe }
	\label{tabela6}
	\def\arraystretch{1.5}
	\begin{tabular}{|p{7.25cm}|>{\centering\arraybackslash}p{7.25cm}|}
		\hline
		\textbf{Classe}                         & \textbf{Porcentagem de acertos} \\ \hline
		\textbf{Conjunto original de 5 classes} & 60.64\%                         \\ \hline
		\textbf{Conjunto com 2 classes}       & 80.14\%                         \\ \hline
	\end{tabular}
\end{table}

%inserir Tabela 7

\begin{table}[h]
	\centering
	\caption{Aplicação do NaïveBayes para os diferentes conjuntos de classe }
	\label{tabela7}
	\def\arraystretch{1.5}
	\begin{tabular}{|p{7.25cm}|>{\centering\arraybackslash}p{7.25cm}|}
		\hline
		\textbf{Classe}                         & \textbf{Porcentagem de acertos} \\ \hline
		\textbf{Conjunto original de 5 classes} & 61.12\%                         \\ \hline
		\textbf{Conjunto com 2 classes}       & 85.62\%                         \\ \hline
	\end{tabular}
\end{table}

\begin{table}[h]
	\centering
	\caption{Aplicação exaustiva do J48}
	\label{tabela7.1}
	\def\arraystretch{2}
	
	\begin{tabular}{|>{\centering\arraybackslash}p{3cm}|>{\centering\arraybackslash}p{5.75cm}|>{\centering\arraybackslash}p{5.75cm}|}
		\hline
		\parbox[l][1.5cm][c]{3cm}{\textbf{Número de \\características}} &
		\parbox[l][1.5cm][c]{5.75cm}{\textbf{\% de acertos - conjunto \\original de 5 classes}} &
		\parbox[l][1.5cm][c]{5.75cm}{\textbf{\% de acertos - conjunto \\de 2 classes}} \\ \hline
		2                                                                                                    & 60,64                                                                                                                                        & 80,14                                                                                                                               \\ \hline
		3                                                                                                    & 55,38                                                                                                                                        & 78,33                                                                                                                               \\ \hline
		4                                                                                                    & 58,55                                                                                                                                        & 77,67                                                                                                                               \\ \hline
		5                                                                                                    & 57,14                                                                                                                                        & 77,5                                                                                                                                \\ \hline
		6                                                                                                    & 56,31                                                                                                                                        & 77,02                                                                                                                               \\ \hline
		7                                                                                                    & 53,45                                                                                                                                        & 76,52                                                                                                                               \\ \hline
		8                                                                                                    & 51,88                                                                                                                                        & 75,88                                                                                                                               \\ \hline
		9                                                                                                    & 51,07                                                                                                                                        & 75,24                                                                                                                               \\ \hline
		10                                                                                                   & 50,6                                                                                                                                         & 74,74                                                                                                                               \\ \hline
		11                                                                                                   & 49,29                                                                                                                                        & 74,74                                                                                                                               \\ \hline
		12                                                                                                   & 48,64                                                                                                                                        & 74,74                                                                                                                               \\ \hline
		13                                                                                                   & 48,81                                                                                                                                        & 74,74                                                                                                                               \\ \hline
		14                                                                                                   & 48,14                                                                                                                                        & 74,74                                                                                                                               \\ \hline
		15                                                                                                   & 48,48                                                                                                                                        & 74,6                                                                                                                                \\ \hline
		16                                                                                                   & 48,48                                                                                                                                        & 74,6                                                                                                                                \\ \hline
	\end{tabular}
\end{table}

\begin{table}[h]
	\centering
	\caption{Aplicação exaustiva do NaïveBayes}
	\label{tabela7.2}
	\def\arraystretch{2}
	
	\begin{tabular}{|>{\centering\arraybackslash}p{3cm}|>{\centering\arraybackslash}p{5.75cm}|>{\centering\arraybackslash}p{5.75cm}|}
		\hline
		\parbox[l][1.5cm][c]{3cm}{\textbf{Número de \\características}} &
		\parbox[l][1.5cm][c]{5.75cm}{\textbf{\% de acertos - conjunto \\original de 5 classes}} &
		\parbox[l][1.5cm][c]{5.75cm}{\textbf{\% de acertos - conjunto \\de 2 classes}} \\ \hline
		2                                                                                                    & 60,02                                                                                                                                        & 80,81                                                                                                                               \\ \hline
		3                                                                                                    & 57,19                                                                                                                                        & 80,95                                                                                                                               \\ \hline
		4                                                                                                    & 55,79                                                                                                                                        & 82,38                                                                                                                               \\ \hline
		5                                                                                                    & 57,88                                                                                                                                        & 82,83                                                                                                                               \\ \hline
		6                                                                                                    & 59,52                                                                                                                                        & 85,62                                                                                                                               \\ \hline
		7                                                                                                    & 58,81                                                                                                                                        & 84,69                                                                                                                               \\ \hline
		8                                                                                                    & 61,12                                                                                                                                        & 85,21                                                                                                                               \\ \hline
		9                                                                                                    & 58,5                                                                                                                                         & 82,55                                                                                                                               \\ \hline
		10                                                                                                   & 58,52                                                                                                                                        & 82,55                                                                                                                               \\ \hline
		11                                                                                                   & 59,21                                                                                                                                        & 82,21                                                                                                                               \\ \hline
		12                                                                                                   & 58,71                                                                                                                                        & 82,55                                                                                                                               \\ \hline
		13                                                                                                   & 59,02                                                                                                                                        & 83,55                                                                                                                               \\ \hline
		14                                                                                                   & 59,48                                                                                                                                        & 83,88                                                                                                                               \\ \hline
		15                                                                                                   & 58,69                                                                                                                                        & 84,17                                                                                                                               \\ \hline
		16                                                                                                   & 58,57                                                                                                                                        & 83,52                                                                                                                               \\ \hline
	\end{tabular}
\end{table}


